\documentclass[12pt,a4paper]{scrartcl}\usepackage[]{graphicx}\usepackage[]{color}
%% maxwidth is the original width if it is less than linewidth
%% otherwise use linewidth (to make sure the graphics do not exceed the margin)
\makeatletter
\def\maxwidth{ %
  \ifdim\Gin@nat@width>\linewidth
    \linewidth
  \else
    \Gin@nat@width
  \fi
}
\makeatother

\definecolor{fgcolor}{rgb}{0.345, 0.345, 0.345}
\newcommand{\hlnum}[1]{\textcolor[rgb]{0.686,0.059,0.569}{#1}}%
\newcommand{\hlstr}[1]{\textcolor[rgb]{0.192,0.494,0.8}{#1}}%
\newcommand{\hlcom}[1]{\textcolor[rgb]{0.678,0.584,0.686}{\textit{#1}}}%
\newcommand{\hlopt}[1]{\textcolor[rgb]{0,0,0}{#1}}%
\newcommand{\hlstd}[1]{\textcolor[rgb]{0.345,0.345,0.345}{#1}}%
\newcommand{\hlkwa}[1]{\textcolor[rgb]{0.161,0.373,0.58}{\textbf{#1}}}%
\newcommand{\hlkwb}[1]{\textcolor[rgb]{0.69,0.353,0.396}{#1}}%
\newcommand{\hlkwc}[1]{\textcolor[rgb]{0.333,0.667,0.333}{#1}}%
\newcommand{\hlkwd}[1]{\textcolor[rgb]{0.737,0.353,0.396}{\textbf{#1}}}%
\let\hlipl\hlkwb

\usepackage{framed}
\makeatletter
\newenvironment{kframe}{%
 \def\at@end@of@kframe{}%
 \ifinner\ifhmode%
  \def\at@end@of@kframe{\end{minipage}}%
  \begin{minipage}{\columnwidth}%
 \fi\fi%
 \def\FrameCommand##1{\hskip\@totalleftmargin \hskip-\fboxsep
 \colorbox{shadecolor}{##1}\hskip-\fboxsep
     % There is no \\@totalrightmargin, so:
     \hskip-\linewidth \hskip-\@totalleftmargin \hskip\columnwidth}%
 \MakeFramed {\advance\hsize-\width
   \@totalleftmargin\z@ \linewidth\hsize
   \@setminipage}}%
 {\par\unskip\endMakeFramed%
 \at@end@of@kframe}
\makeatother

\definecolor{shadecolor}{rgb}{.97, .97, .97}
\definecolor{messagecolor}{rgb}{0, 0, 0}
\definecolor{warningcolor}{rgb}{1, 0, 1}
\definecolor{errorcolor}{rgb}{1, 0, 0}
\newenvironment{knitrout}{}{} % an empty environment to be redefined in TeX

\usepackage{alltt}
\usepackage[utf8]{inputenc}
\usepackage{amsmath}
\usepackage{graphicx}
\usepackage{tikz}
%\usepackage{silence}
\usepackage{mdframed}
%\WarningFilter{mdframed}{You got a bad break}
\usepackage[colorinlistoftodos]{todonotes}
\usepackage{listings}
\usepackage{color}
\colorlet{exampcol}{blue!10}
\usepackage{multicol}
\usepackage{booktabs}

\title{A minimalistic introduction to R}
\date{\today}
\author{Timothée Bonnet \& al.}
\IfFileExists{upquote.sty}{\usepackage{upquote}}{}
\begin{document}


\maketitle

\textit{This document recycles tutorials written by Koen van Benthem and Tina Cornioley}

\tableofcontents



\vspace{2cm}
\begin{mdframed}
\textit{There are many ways to achieve the same goal in \texttt{R}, and we do not claim to teach you the most efficient way to use \texttt{R}. If you at some point during the computer practicals encounter a code that you could make more efficient or elegant, please do let us know! \\[1.5ex] 
Do try to understand exactly what the code and the functions we use do. The best way to learn how functions work is by either using the \texttt{R}-manaul (type \texttt{?functionname} or use the RStudio Help tab by clicking on it or pressing F1) or by creating dummy data (just make up a small amount of data yourself, using \texttt{R} if possible!) and analyse what the function does to this data.}
\end{mdframed}
\newpage

\section*{How this document works}
This is a \texttt{knitr} document, which knits \texttt{R} code and output within a \LaTeX document.
R code and output is generally contained within boxes with a gray background. Comments within the R code start with a \# symbol; lines with R-outputs start with \#\#.

All the files necessary to go through the workshop are (or should be!) in the folder of a github repository. We recommand you copy these files, or fork the repository if you are a git user.

Now, let's the fun begin.

\section{Trash your calculator}
\subsection{Operators}
\texttt{R} can be used as a calculator, and a far more powerful one that any physical calculator. If you use your calculator to enter numbers in \texttt{R}, you are being inefficient.

Below we demonstrate the use of some basic mathematical operators:
\begin{knitrout}
\definecolor{shadecolor}{rgb}{0.969, 0.969, 0.969}\color{fgcolor}\begin{kframe}
\begin{alltt}
\hlnum{1}\hlopt{+}\hlnum{3} \hlcom{#addition}
\end{alltt}
\begin{verbatim}
## [1] 4
\end{verbatim}
\begin{alltt}
\hlnum{5}\hlopt{-}\hlnum{2} \hlcom{#substraction}
\end{alltt}
\begin{verbatim}
## [1] 3
\end{verbatim}
\begin{alltt}
\hlnum{6}\hlopt{*}\hlnum{4} \hlcom{#multiplication}
\end{alltt}
\begin{verbatim}
## [1] 24
\end{verbatim}
\begin{alltt}
\hlnum{14}\hlopt{/}\hlnum{2} \hlcom{#division}
\end{alltt}
\begin{verbatim}
## [1] 7
\end{verbatim}
\begin{alltt}
\hlnum{2}\hlopt{^}\hlnum{3} \hlcom{#exponent}
\end{alltt}
\begin{verbatim}
## [1] 8
\end{verbatim}
\begin{alltt}
\hlnum{2}\hlopt{**}\hlnum{3} \hlcom{#or equivalently}
\end{alltt}
\begin{verbatim}
## [1] 8
\end{verbatim}
\end{kframe}
\end{knitrout}

There are many mathematical functions already present in R:
\begin{knitrout}
\definecolor{shadecolor}{rgb}{0.969, 0.969, 0.969}\color{fgcolor}\begin{kframe}
\begin{alltt}
\hlkwd{exp}\hlstd{(}\hlnum{3}\hlstd{)} \hlcom{#exponential}
\end{alltt}
\begin{verbatim}
## [1] 20.08554
\end{verbatim}
\begin{alltt}
\hlkwd{log}\hlstd{(}\hlnum{2.71}\hlstd{)} \hlcom{#logarithm}
\end{alltt}
\begin{verbatim}
## [1] 0.9969486
\end{verbatim}
\begin{alltt}
\hlkwd{sqrt}\hlstd{(}\hlnum{9}\hlstd{)} \hlcom{#square root, which of course you can also write as:}
\end{alltt}
\begin{verbatim}
## [1] 3
\end{verbatim}
\begin{alltt}
\hlnum{9} \hlopt{^} \hlstd{(}\hlnum{1}\hlopt{/}\hlnum{2}\hlstd{)}
\end{alltt}
\begin{verbatim}
## [1] 3
\end{verbatim}
\begin{alltt}
\hlkwd{sin}\hlstd{(pi}\hlopt{/}\hlnum{2}\hlstd{);} \hlkwd{cos}\hlstd{(}\hlnum{1}\hlstd{);} \hlkwd{tan}\hlstd{(pi}\hlopt{/}\hlnum{3}\hlstd{)} \hlcom{#trigonometric functions}
\end{alltt}
\begin{verbatim}
## [1] 1
## [1] 0.5403023
## [1] 1.732051
\end{verbatim}
\end{kframe}
\end{knitrout}

\begin{mdframed}{Small}
Use R to compute
$$
  y = \frac{1}{2\sqrt{2\pi}} e^{\frac{-1}{2} (\frac{3-\pi}{2})^2}
$$



\end{mdframed}

Logical operators are very important for programming and scripting.
You can test whether two things are equal with double = signs:
\begin{knitrout}
\definecolor{shadecolor}{rgb}{0.969, 0.969, 0.969}\color{fgcolor}\begin{kframe}
\begin{alltt}
\hlnum{3} \hlopt{==} \hlnum{6}\hlopt{/}\hlnum{2} \hlcom{#is 3 equal to 6/2? TRUE!}
\end{alltt}
\begin{verbatim}
## [1] TRUE
\end{verbatim}
\begin{alltt}
\hlnum{3} \hlopt{==} \hlstd{pi}  \hlcom{# FALSE!}
\end{alltt}
\begin{verbatim}
## [1] FALSE
\end{verbatim}
\end{kframe}
\end{knitrout}

You can also test if they are NOT equal with the operator \texttt{!=}:
\begin{knitrout}
\definecolor{shadecolor}{rgb}{0.969, 0.969, 0.969}\color{fgcolor}\begin{kframe}
\begin{alltt}
\hlnum{2} \hlopt{!=} \hlnum{3}
\end{alltt}
\begin{verbatim}
## [1] TRUE
\end{verbatim}
\begin{alltt}
\hlnum{2} \hlopt{!=} \hlnum{2}
\end{alltt}
\begin{verbatim}
## [1] FALSE
\end{verbatim}
\end{kframe}
\end{knitrout}

The AND operator is \texttt{\&}
\begin{knitrout}
\definecolor{shadecolor}{rgb}{0.969, 0.969, 0.969}\color{fgcolor}\begin{kframe}
\begin{alltt}
\hlnum{2} \hlopt{==} \hlnum{2} \hlopt{&} \hlnum{3}\hlopt{==}\hlnum{3}
\end{alltt}
\begin{verbatim}
## [1] TRUE
\end{verbatim}
\begin{alltt}
\hlnum{2} \hlopt{==}\hlnum{2} \hlopt{&} \hlnum{3}\hlopt{==}\hlnum{2}
\end{alltt}
\begin{verbatim}
## [1] FALSE
\end{verbatim}
\end{kframe}
\end{knitrout}

The OR operator is \texttt{\textbar}
\begin{knitrout}
\definecolor{shadecolor}{rgb}{0.969, 0.969, 0.969}\color{fgcolor}\begin{kframe}
\begin{alltt}
\hlnum{2} \hlopt{==} \hlnum{2} \hlopt{|} \hlnum{3}\hlopt{==}\hlnum{2}
\end{alltt}
\begin{verbatim}
## [1] TRUE
\end{verbatim}
\begin{alltt}
\hlnum{2} \hlopt{==} \hlnum{4} \hlopt{|} \hlnum{3}\hlopt{==}\hlnum{2}
\end{alltt}
\begin{verbatim}
## [1] FALSE
\end{verbatim}
\end{kframe}
\end{knitrout}



\subsection{Assignment}




\section{loops and if statements}

\subsection{for loops}

\subsection{while loops}

\subsection{if statements}

\end{document}
