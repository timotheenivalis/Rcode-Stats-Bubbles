\documentclass[10pt]{beamer}\usepackage[]{graphicx}\usepackage[]{color}
%% maxwidth is the original width if it is less than linewidth
%% otherwise use linewidth (to make sure the graphics do not exceed the margin)
\makeatletter
\def\maxwidth{ %
  \ifdim\Gin@nat@width>\linewidth
    \linewidth
  \else
    \Gin@nat@width
  \fi
}
\makeatother

\definecolor{fgcolor}{rgb}{0.345, 0.345, 0.345}
\newcommand{\hlnum}[1]{\textcolor[rgb]{0.686,0.059,0.569}{#1}}%
\newcommand{\hlstr}[1]{\textcolor[rgb]{0.192,0.494,0.8}{#1}}%
\newcommand{\hlcom}[1]{\textcolor[rgb]{0.678,0.584,0.686}{\textit{#1}}}%
\newcommand{\hlopt}[1]{\textcolor[rgb]{0,0,0}{#1}}%
\newcommand{\hlstd}[1]{\textcolor[rgb]{0.345,0.345,0.345}{#1}}%
\newcommand{\hlkwa}[1]{\textcolor[rgb]{0.161,0.373,0.58}{\textbf{#1}}}%
\newcommand{\hlkwb}[1]{\textcolor[rgb]{0.69,0.353,0.396}{#1}}%
\newcommand{\hlkwc}[1]{\textcolor[rgb]{0.333,0.667,0.333}{#1}}%
\newcommand{\hlkwd}[1]{\textcolor[rgb]{0.737,0.353,0.396}{\textbf{#1}}}%
\let\hlipl\hlkwb

\usepackage{framed}
\makeatletter
\newenvironment{kframe}{%
 \def\at@end@of@kframe{}%
 \ifinner\ifhmode%
  \def\at@end@of@kframe{\end{minipage}}%
  \begin{minipage}{\columnwidth}%
 \fi\fi%
 \def\FrameCommand##1{\hskip\@totalleftmargin \hskip-\fboxsep
 \colorbox{shadecolor}{##1}\hskip-\fboxsep
     % There is no \\@totalrightmargin, so:
     \hskip-\linewidth \hskip-\@totalleftmargin \hskip\columnwidth}%
 \MakeFramed {\advance\hsize-\width
   \@totalleftmargin\z@ \linewidth\hsize
   \@setminipage}}%
 {\par\unskip\endMakeFramed%
 \at@end@of@kframe}
\makeatother

\definecolor{shadecolor}{rgb}{.97, .97, .97}
\definecolor{messagecolor}{rgb}{0, 0, 0}
\definecolor{warningcolor}{rgb}{1, 0, 1}
\definecolor{errorcolor}{rgb}{1, 0, 0}
\newenvironment{knitrout}{}{} % an empty environment to be redefined in TeX

\usepackage{alltt}%
\usetheme{Boadilla}
\usecolortheme{seahorse}

\usepackage[utf8]{inputenc}%

% graphics
%% Figures %%%%%%%%%%%%%%%%%%%%%%%%%%%%%%%%%%%%%%%%%%%%%%%%%%
\usepackage{graphicx}
\usepackage{xcolor}%for color mixing

\usepackage{amsmath}%
\usepackage{amsfonts}%
\usepackage{amssymb}%
\usepackage{graphicx}

\usepackage{tikz}

%%%%%%%%%%%%%%%%%%%%%%%%%%%%%%%%%%%%%%%%%%%%%%
%%%%%%%%%%%%%%%%% Doc info %%%%%%%%%%%%%%%%%%%
\title[\textbf{Linear models}]{Statistical inference and linear models}
\date{\today}

%%%%%%%%%%%%%%%%%%%%%%%%%%%%%%%%%%%%%%%%%%%%%%
\IfFileExists{upquote.sty}{\usepackage{upquote}}{}
\begin{document}




\begin{frame}
\maketitle	
\end{frame}
%%%%%%%%%%%

\AtBeginSection[]
{
  \begin{frame}<beamer>
    \frametitle{}
    \tableofcontents[currentsection,hideothersubsections,subsectionstyle=hide]% down vote\tableofcontents[currentsection,currentsubsection,hideothersubsections,sectionstyle=show/hide,subsectionstyle=show/shaded/hide] 
  \end{frame}
} 

\begin{frame}{If you get bored}
  \begin{itemize}
    \item Go to the last slide for bonus exercises
    \item Work on code for your research and ask question during exercise time
    \item But try and keep an eye out, just in case
  \end{itemize}
  
  \begin{center}
    \includegraphics[height=0.5\textheight]{Figures/too-easy-whats-next.jpg}
  \end{center}
\end{frame}
%%%%%%%%%%%

\begin{frame}{Disclaimer}
  \begin{itemize}
    \item Assume you got lectures about statistics and
       \begin{itemize}
        \item know why we need statistics
        \item have heard of the general philosophy
       \end{itemize}
    \item We may simplify to focus on practical aspects
    \item Correct us if we say something completely awful
  \end{itemize}
  
\end{frame}
%%%%%%%%%%%

\section{Statistical inference}%very general, still simplistic

\begin{frame}{General approach}

\begin{center}
  \begin{tikzpicture}
    \node (sq) at (0,-1) {\color{red}{1. Scientific question}};
    \pause
    \node (mo) at (0,-2) {2. Model and Statistical question};
    \draw[->, thick] (sq)--(mo);
    \pause
    \node (dac) at (6,-2) {\color{red}{3. Data collection}};
    \draw[<->, thick] (mo)--(dac);
    \pause
    \node (est) at (0,-3) {4. Estimation};
        \draw[->, thick] (mo)--(est);
    \node (unc) at (0,-3.5) {4.b Uncertainty and statistical significance};
    \pause
    
    \node (che) at (0,-5) {5. Diagnostic, check assumptions, prediction};
        \draw[->, thick] (unc)--(che);
    \draw[->, thick] (che.west) to [out=150, in=210] (mo.west);

    \pause
    \node (int) at (0,-6) {\color{red}{6. Interpret and think about the biology}};
        \draw[->, thick] (che)--(int);

  \draw[rounded corners, color=blue] (-4.5,-1.5) rectangle (4,-5.5);
  \node[anchor=north west] (r) at (-4.5,-1.5) {\includegraphics[width=0.1\textwidth]{Figures/r}};
  \end{tikzpicture}
  \end{center}
\end{frame}
%%%%%%%%%%%

\begin{frame}[fragile]{Reminder t.test}
  
  \pause
\begin{knitrout}
\definecolor{shadecolor}{rgb}{0.843, 0.867, 0.922}\color{fgcolor}\begin{kframe}
\begin{alltt}
  \hlkwd{data}\hlstd{(}\hlstr{"iris"}\hlstd{)}
\end{alltt}
\end{kframe}
\end{knitrout}

\begin{knitrout}
\definecolor{shadecolor}{rgb}{0.843, 0.867, 0.922}\color{fgcolor}\begin{kframe}
\begin{alltt}
  \hlkwd{str}\hlstd{(iris)}
  \hlkwd{plot}\hlstd{(iris)}
\end{alltt}
\end{kframe}
\end{knitrout}
  
  \begin{enumerate}
    \item \color{red}{Scientific question}: Are the taxa "setosa" and "versicolor" different species?
  \pause
    \item Model and stat question:
      \begin{itemize}
        \item Model:
          \begin{itemize}
            \item There is an intrinsic/expected sepal length value for a species; an individual value is the sum of this expectation and a random Gaussian deviation.
            \item $y_i = \mu_{species_i} + \epsilon_i$ with  $\epsilon \sim N(0,\sigma^2)$
            \item t-test
          \end{itemize}
        \pause
        \item Statistical question: 
          \begin{itemize}
            \item Does sepal length \textbf{differ significantly} between the two taxa \textbf{in our sample}?
            \item Is the observed difference between taxa likely if both taxa have the same intrinsic/expected value?
          \end{itemize}
        \end{itemize}
      \item Data collection
    \end{enumerate}
\end{frame}
%%%%%%%%%%%

\begin{frame}[fragile]{Reminder t.test}
    One t-test for sepal length between \textit{setosa} and \textit{versicolor}:
\begin{knitrout}
\definecolor{shadecolor}{rgb}{0.843, 0.867, 0.922}\color{fgcolor}\begin{kframe}
\begin{alltt}
  \hlkwd{t.test}\hlstd{(}\hlkwc{x} \hlstd{= iris}\hlopt{$}\hlstd{Sepal.Length[iris}\hlopt{$}\hlstd{Species} \hlopt{==} \hlstr{"setosa"}\hlstd{],}
        \hlkwc{y} \hlstd{= iris}\hlopt{$}\hlstd{Sepal.Length[iris}\hlopt{$}\hlstd{Species} \hlopt{==} \hlstr{"versicolor"}\hlstd{])}
\end{alltt}
\begin{verbatim}

	Welch Two Sample t-test

data:  iris$Sepal.Length[iris$Species == "setosa"] and iris$Sepal.Length[iris$Species == "versicolor"]
t = -10.521, df = 86.538, p-value < 2.2e-16
alternative hypothesis: true difference in means is not equal to 0
95 percent confidence interval:
 -1.1057074 -0.7542926
sample estimates:
mean of x mean of y 
    5.006     5.936 
\end{verbatim}
\end{kframe}
\end{knitrout}
  
\end{frame}
%%%%%%%%%%%

% \begin{frame}[fragile]{Reminder t.test: Are they different?}
% 
%   <<boxplot, fig.width=5, fig.height=5, out.width='0.5\\textwidth', eval=FALSE>>=
%   boxplot(Sepal.Length ~ Species, 
%           data = iris[iris$Species %in% c("setosa","versicolor"),],
%           drop = TRUE, ylab="Sepal length", xlab="Species")
%   @
% \begin{columns}
% \begin{column}{0.5\textwidth}
% <<boxplot2, fig.width=5, fig.height=5, out.width='0.9\\textwidth', echo=FALSE>>=
%   boxplot(Sepal.Length ~ Species, 
%           data = iris[iris$Species %in% c("setosa","versicolor"),],
%           drop = TRUE, ylab="Sepal length", xlab="Species")
%   @
% \end{column}
% \begin{column}{0.5\textwidth}
%   \begin{itemize}
%     \item Means: 5.006 vs. 5.936
%     \item Standard deviation: 0.35 and 0.52
%     \item Standard error (SD/ $\sqrt n$): 0.05 and 0.07
%   \end{itemize}
% \end{column}
% \end{columns}
% \end{frame}
% %%%%%%%%%%%

\begin{frame}[fragile]{When do we know it is different?}

\begin{enumerate}
  \setcounter{enumi}{3}
  \item Statistical estimation
  \begin{itemize}
    \item a Estimation
      \begin{itemize}
        \item Cannot know true difference $\mu_{species_1} - \mu_{species_2}$
        \item Estimated difference $= \color{red}{\text{Mean}_1 - \text{Mean}_2 }$
        \item Difference contains random variation
      \end{itemize}
    \item b Quantify uncertainty / Statistical significance
      \begin{itemize}
        \item $
      t = \frac{\color{red}{\text{Mean}_1 - \text{Mean}_2}}{\text{\color{orange}Variation}}
      \frac{\sqrt{{\text{\color{blue}{Sample Size}}}}}{\sqrt{2}}
      $
        \item We know exactly how t is distributed when $\mu_{species_1} - \mu_{species_2} = 0$
        \item Hence we know probability of $\geq t$ if $\mu_{species_1} - \mu_{species_2} = 0$
        \item Can derive confidence interval and standard error
      \end{itemize}
  \end{itemize}
\end{enumerate}

\pause
Less uncertainty with
  \begin{itemize}
    \item \color{red}{Larger absolute difference}
    \item \color{orange}{Smaller variability}
    \item \color{blue}{Larger sample size}
  \end{itemize}


\end{frame}
%%%%%%%%%%%
  
\begin{frame}[fragile]{When do we know it is different?}
\textbf{\color{red}{1. Larger absolute difference}}
\begin{knitrout}
\definecolor{shadecolor}{rgb}{0.843, 0.867, 0.922}\color{fgcolor}\begin{kframe}
\begin{alltt}
\hlstd{nbsim} \hlkwb{<-} \hlnum{1000}
\hlstd{pdistri_large} \hlkwb{<-} \hlkwd{vector}\hlstd{(}\hlkwc{length} \hlstd{= nbsim)}
\hlstd{pdistri_small} \hlkwb{<-} \hlkwd{vector}\hlstd{(}\hlkwc{length} \hlstd{= nbsim)}
\hlkwa{for} \hlstd{(i} \hlkwa{in} \hlnum{1}\hlopt{:}\hlstd{nbsim)}
  \hlstd{\{}
  \hlstd{x1} \hlkwb{<-} \hlkwd{rnorm}\hlstd{(}\hlkwc{n} \hlstd{=} \hlnum{10}\hlstd{,} \hlkwc{mean} \hlstd{=} \hlnum{2}\hlstd{,} \hlkwc{sd} \hlstd{=} \hlnum{1}\hlstd{)}
  \hlstd{x2} \hlkwb{<-} \hlkwd{rnorm}\hlstd{(}\hlkwc{n} \hlstd{=} \hlnum{10}\hlstd{,} \hlkwc{mean} \hlstd{=} \hlnum{4}\hlstd{,} \hlkwc{sd} \hlstd{=} \hlnum{1}\hlstd{)} \hlcom{#large diff}
  \hlstd{x3} \hlkwb{<-} \hlkwd{rnorm}\hlstd{(}\hlkwc{n} \hlstd{=} \hlnum{10}\hlstd{,} \hlkwc{mean} \hlstd{=} \hlnum{2.5}\hlstd{,} \hlkwc{sd} \hlstd{=} \hlnum{1}\hlstd{)} \hlcom{#small diff}
  \hlstd{out_large} \hlkwb{<-} \hlkwd{t.test}\hlstd{(x1, x2)}
  \hlstd{out_small} \hlkwb{<-} \hlkwd{t.test}\hlstd{(x1, x3)}
  \hlstd{pdistri_large[i]}\hlkwb{<-}\hlstd{out_large}\hlopt{$}\hlstd{p.value}
  \hlstd{pdistri_small[i]}\hlkwb{<-}\hlstd{out_small}\hlopt{$}\hlstd{p.value}
\hlstd{\}}
\end{alltt}
\end{kframe}
\end{knitrout}
\end{frame}
%%%%%%%%%%
\begin{frame}[fragile]{When do we know it is different?}
\centering
\begin{knitrout}
\definecolor{shadecolor}{rgb}{0.843, 0.867, 0.922}\color{fgcolor}\begin{kframe}
\begin{alltt}
\hlkwd{par}\hlstd{(}\hlkwc{mfrow}\hlstd{=}\hlkwd{c}\hlstd{(}\hlnum{1}\hlstd{,}\hlnum{2}\hlstd{),} \hlkwc{cex}\hlstd{=}\hlnum{2}\hlstd{)}
\hlkwd{hist}\hlstd{(pdistri_large,} \hlkwc{xlim}\hlstd{=}\hlkwd{c}\hlstd{(}\hlnum{0}\hlstd{,}\hlnum{1}\hlstd{),}
     \hlkwc{main}\hlstd{=}\hlkwd{paste}\hlstd{(}\hlstr{"Prop signif="}\hlstd{,}\hlkwd{mean}\hlstd{(pdistri_large}\hlopt{<}\hlnum{0.05}\hlstd{)))}
\hlkwd{hist}\hlstd{(pdistri_small,} \hlkwc{xlim}\hlstd{=}\hlkwd{c}\hlstd{(}\hlnum{0}\hlstd{,}\hlnum{1}\hlstd{),}
     \hlkwc{main}\hlstd{=}\hlkwd{paste}\hlstd{(}\hlstr{"Prop signi="}\hlstd{,}\hlkwd{mean}\hlstd{(pdistri_small}\hlopt{<}\hlnum{0.05}\hlstd{)))}
\end{alltt}
\end{kframe}
\includegraphics[width=0.8\textwidth,height=0.5\textheight]{figure/comphist1-1} 
\begin{kframe}\begin{alltt}
\hlkwd{par}\hlstd{(}\hlkwc{mfrow}\hlstd{=}\hlkwd{c}\hlstd{(}\hlnum{1}\hlstd{,}\hlnum{1}\hlstd{))}
\end{alltt}
\end{kframe}
\end{knitrout}

\end{frame}
%%%%%%%%%%%

\begin{frame}[fragile]{When do we know it is different? Try it!}

\begin{alertblock}{Exercise}
Check the effect of {\color{orange}{smaller variability}} and/or {\color{blue}{larger sample size}}.
\end{alertblock}
\end{frame}
%%%%%%%%%%%

\begin{frame}{By the way, what are these p-values?}
Blabla

\begin{exampleblock}{Properties}
  \begin{itemize}
    \item Reference to a null-model ($H_0$) with assumptions
    \item Uniform distribution under $H_0$ \dots
    \item \dots hence proportion(significance under $H_0$) = significance threshold
  \end{itemize}
\end{exampleblock}
\end{frame}
%%%%%%%%%%%

\begin{frame}[fragile]{T-test exercise}

\begin{knitrout}
\definecolor{shadecolor}{rgb}{0.843, 0.867, 0.922}\color{fgcolor}\begin{kframe}
\begin{alltt}
\hlkwd{t.test}\hlstd{(}\hlkwc{x} \hlstd{= ...,} \hlkwc{y}\hlstd{=....,} \hlkwc{var.equal} \hlstd{=} \hlnum{TRUE}\hlstd{)}
\hlkwd{t.test}\hlstd{(}\hlkwc{x} \hlstd{= ...,} \hlkwc{y}\hlstd{=....,} \hlkwc{var.equal} \hlstd{=} \hlnum{FALSE}\hlstd{)}
\end{alltt}
\end{kframe}
\end{knitrout}

What if variance are different by chance only?
\begin{knitrout}
\definecolor{shadecolor}{rgb}{0.843, 0.867, 0.922}\color{fgcolor}\begin{kframe}
\begin{alltt}
\hlkwd{set.seed}\hlstd{(}\hlnum{1234}\hlstd{)}
\hlkwd{var}\hlstd{(}\hlkwd{rnorm}\hlstd{(}\hlnum{20}\hlstd{,} \hlkwc{mean} \hlstd{=} \hlnum{0}\hlstd{,} \hlkwc{sd} \hlstd{=} \hlnum{1}\hlstd{))}
\end{alltt}
\begin{verbatim}
[1] 1.027806
\end{verbatim}
\begin{alltt}
\hlkwd{var}\hlstd{(}\hlkwd{rnorm}\hlstd{(}\hlnum{20}\hlstd{,} \hlkwc{mean} \hlstd{=} \hlnum{0}\hlstd{,} \hlkwc{sd} \hlstd{=} \hlnum{1}\hlstd{))}
\end{alltt}
\begin{verbatim}
[1] 0.6265501
\end{verbatim}
\end{kframe}
\end{knitrout}

\begin{alertblock}{Exercise}
  What option is more correct for var.equal?
\end{alertblock}

\end{frame}
%%%%%%%%%%%
\section{t-test, ANOVA, regression: all is one, one is all}

\begin{frame}[fragile]{A small example}

Animal behavior in response to weather



Load data:
\begin{knitrout}
\definecolor{shadecolor}{rgb}{0.843, 0.867, 0.922}\color{fgcolor}\begin{kframe}
\begin{alltt}
\hlkwd{getwd}\hlstd{()}
\hlkwd{setwd}\hlstd{()}
\end{alltt}
\end{kframe}
\end{knitrout}

\begin{knitrout}
\definecolor{shadecolor}{rgb}{0.843, 0.867, 0.922}\color{fgcolor}\begin{kframe}
\begin{alltt}
\hlstd{dat.behav} \hlkwb{<-} \hlkwd{read.csv}\hlstd{(}\hlkwc{file} \hlstd{=} \hlstr{"datbehav.csv"}\hlstd{)} \hlcom{# path to file}
\end{alltt}
\end{kframe}
\end{knitrout}

\pause
STEP 1: have a look at your data
\begin{knitrout}
\definecolor{shadecolor}{rgb}{0.843, 0.867, 0.922}\color{fgcolor}\begin{kframe}
\begin{alltt}
\hlkwd{str}\hlstd{(dat.behav)}
\hlkwd{summary}\hlstd{(dat.behav)}
\hlkwd{plot}\hlstd{(dat.behav)}
\end{alltt}
\end{kframe}
\end{knitrout}
\end{frame}
%%%%%%%%%%%%%%%%%%%%%%

\begin{frame}[fragile]{t-test}
\begin{knitrout}
\definecolor{shadecolor}{rgb}{0.843, 0.867, 0.922}\color{fgcolor}\begin{kframe}
\begin{alltt}
\hlstd{fitstudent} \hlkwb{<-} \hlkwd{t.test}\hlstd{(}\hlkwc{x} \hlstd{= dat.behav}\hlopt{$}\hlstd{activity[dat.behav}\hlopt{$}\hlstd{weather}\hlopt{==}
                                              \hlstr{"rainy"}\hlstd{],}
                     \hlkwc{y} \hlstd{= dat.behav}\hlopt{$}\hlstd{activity[dat.behav}\hlopt{$}\hlstd{weather}\hlopt{==}
                                              \hlstr{"sunny"}\hlstd{],}
                     \hlkwc{var.equal} \hlstd{=} \hlnum{TRUE}\hlstd{)}
\hlkwd{print}\hlstd{(fitstudent)}
\end{alltt}
\begin{verbatim}

	Two Sample t-test

data:  dat.behav$activity[dat.behav$weather == "rainy"] and dat.behav$activity[dat.behav$weather == "sunny"]
t = 3.2752, df = 33, p-value = 0.002485
alternative hypothesis: true difference in means is not equal to 0
95 percent confidence interval:
 0.6138373 2.6270325
sample estimates:
mean of x mean of y 
 6.781476  5.161041 
\end{verbatim}
\end{kframe}
\end{knitrout}
\end{frame}
%%%%%%%%%%%

\begin{frame}[fragile]{ANOVA}


\begin{knitrout}
\definecolor{shadecolor}{rgb}{0.843, 0.867, 0.922}\color{fgcolor}\begin{kframe}
\begin{alltt}
\hlstd{fitanova} \hlkwb{<-} \hlkwd{aov}\hlstd{(}\hlkwc{data} \hlstd{= dat.behav,} \hlkwc{formula} \hlstd{= activity} \hlopt{~} \hlstd{weather)}
\hlkwd{summary}\hlstd{(fitanova)}
\end{alltt}
\begin{verbatim}
            Df Sum Sq Mean Sq F value  Pr(>F)   
weather      1  11.25  11.253   10.73 0.00248 **
Residuals   33  34.62   1.049                   
---
Signif. codes:  0 '***' 0.001 '**' 0.01 '*' 0.05 '.' 0.1 ' ' 1
\end{verbatim}
\end{kframe}
\end{knitrout}
\end{frame}
%%%%%%%%%%%

\begin{frame}[fragile]{Linear regression}

\begin{knitrout}
\definecolor{shadecolor}{rgb}{0.843, 0.867, 0.922}\color{fgcolor}\begin{kframe}
\begin{alltt}
\hlstd{fitlm} \hlkwb{<-} \hlkwd{lm}\hlstd{(}\hlkwc{data} \hlstd{= dat.behav,} \hlkwc{formula} \hlstd{= activity} \hlopt{~} \hlstd{weather)}
\hlkwd{summary}\hlstd{(fitlm)}
\end{alltt}
\begin{verbatim}

Call:
lm(formula = activity ~ weather, data = dat.behav)

Residuals:
    Min      1Q  Median      3Q     Max 
-2.3547 -0.6028  0.2346  0.6419  1.6534 

Coefficients:
             Estimate Std. Error t value Pr(>|t|)    
(Intercept)    6.7815     0.4581  14.805 3.94e-16 ***
weathersunny  -1.6204     0.4948  -3.275  0.00248 ** 
---
Signif. codes:  0 '***' 0.001 '**' 0.01 '*' 0.05 '.' 0.1 ' ' 1

Residual standard error: 1.024 on 33 degrees of freedom
Multiple R-squared:  0.2453,	Adjusted R-squared:  0.2224 
F-statistic: 10.73 on 1 and 33 DF,  p-value: 0.002485
\end{verbatim}
\end{kframe}
\end{knitrout}
\end{frame}
%%%%%%%%%%%

\begin{frame}[fragile]{NB: aov() vs. anova()}

\begin{knitrout}
\definecolor{shadecolor}{rgb}{0.843, 0.867, 0.922}\color{fgcolor}\begin{kframe}
\begin{alltt}
\hlkwd{aov}\hlstd{(}\hlkwc{data} \hlstd{= dat.behav,} \hlkwc{formula} \hlstd{= activity} \hlopt{~} \hlstd{weather)}
\hlkwd{anova}\hlstd{(fitlm)}
\end{alltt}
\end{kframe}
\end{knitrout}

\end{frame}
%%%%%%%%%%%


%
% \begin{frame}{Visualizing}
% \begin{figure}
% <<boxplot2, dev='tikz', echo=FALSE >>=
% setPar()
% boxplot(activity ~ weather, data = dat.behav, ylab="activity")
% @
% \end{figure}
% \end{frame}
% %%%%%%%%%%%%%%%%%%%%%%
%
% \begin{frame}{Does wheather change behavior?}
%
% <<>>=
% tt <- t.test(formula = activity ~ weather, data = dat.behav, var.equal = TRUE)
%
% @
%
% \end{frame}
% %%%%%%%%%%%


\end{document}
